WebGL enables a new generation of web applications which have access to hardware-accelerated 3D graphics.
This has impact in the area of real-time applications such as computer games, which are seeing an increasingly large focus on the web.

A key challenge in making 3D games for web browsers is the creation and transmission of assets including models and textures.
Large meshes and textures, characteristic of most modern AAA games, present a great challenge for web games, where load times are typically very small.

Procedural content generation involves the generation of assets using computer algorithms.
The benefit of doing this is that assets do not have to be physically stored or loaded, but are generated by the application itself, minimizing program size.
Games such as .kkrieger do this successfully to create scenes similar in quality to modern titles such as Doom 3, with a very small executable size.

This dissertation focuses on providing a general solution for how procedural content generation can be used to replace the traditional asset pipeline.
The design concentrates on how to provide procedurally generated geometry and textures in place of static assets.
3D geometric information of the areas of a building are generated using 2D procedural plan generation with extrusion into 3D.
Textures are generated using procedural techniques such as bump mapping and Perlin noise.
A general design is presented in detail, along with a prototype which presents an implementation of this design.
This prototype looks at how to generate indoor environments for WebGL using this design.

The prototype allows us to evaluate procedural content generation against static assets in a number of key areas such as initialisation time, aesthetic quality, and ease of creation.
We then present a detailed analysis of the design and how it can be applied in other areas of restricted computational power such as embedded devices.
